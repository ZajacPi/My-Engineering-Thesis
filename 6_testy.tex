\chapter{Testy}
\label{cha:testy}

\section{Działanie i testy robota}
\subsection{Sprawdzenie efektywności momentu siły pierwszej osi}
Po sprawdzeniu czy każda z osi poprawnie się porusza (nie wykryto żadnych oporów ruchu) eksperymentalnie sprawdzono momenty sił zawieszając obciążenia na robocie.

\subsection{Sterowanie silnika krokowego}
Stworzono płytkę prototypową z modułem DRV8855 i ESP32. 
Test1: 
Sterowanie silnika krokowego przy pomocy ESP32 zakończone sukcesem - płytka działa. 

Test2:
Sterowanie przez microROS.

\subsection{Sterowanie głosowe}
- Przykłady komend, czas reakcji, skuteczność rozpoznawania.  

\subsection{Analiza wyników}
- Testy poprawności działania.  
- Ocena stabilności manipulatora.  
- Ograniczenia projektu.  

\section{Możliwości dalszego rozwoju}
\subsection{Implementacja systemów wizyjnych (np śledzenie obiektu)}
\subsection{Sterowanie z poziomu aplikacji webowej}
\subsection{Integracja z dodatkowymi czujnikami}
\subsection{Rozszerzenie o systemy sztucznej inteligencji}