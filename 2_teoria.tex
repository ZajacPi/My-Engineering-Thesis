\chapter{Podstawy Teoretyczne}
\label{cha:podstawyTeoretyczne}



\section{Aspekty prawne i bezpieczeństwo robotów wykonawczych}
\label{sec:kompilacja}
- technologia open-source
Projektu nie udałoby się wykonać, gdyby nie bezinteresowna praca wielu ludzi działających w projektach open source. 

- Normy bezpieczeństwa (np. ISO 10218).  
- Zagrożenia przy pracy robotów z ludźmi.  
- Odpowiedzialność prawna. 


\section{Sterowanie i algorytmy w robotyce}
\label{sec:narzedzia}

- Kinematyka prosta i odwrotna.  
- Sterowanie w przestrzeni współrzędnych.  
- Podstawy ROS2.  

\section{ROS2}
Opisać ROS2 dlaczego go wziąłem, co umożliwia. 
ROS2, czyli Robot Operating System to zestaw bibliotek oraz narzędzi znacznie ułatwiajacych pracę z projektami robotów. 



