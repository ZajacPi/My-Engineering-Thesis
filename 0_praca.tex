\documentclass[12pt]{aghdpl}
% \documentclass[language=en,11pt]{aghdpl}  % praca w języku angielskim

%---------------------------------------------------------------------------
% hyperref dodaje odnośniki wewnętrz dokumentu (np. spis treści jest już klikalny, dotyczy również odwołań, obrazków, cytowań, równań itp.). Pozostałe opcje powodują brak widoczności linków oraz zmieniają właściwości pliku pdf. Szczegóły: https://www.overleaf.com/learn/latex/Hyperlinks
\usepackage[hidelinks, pdftitle={Ramię robota przemysłowego wspomagającego zadania montażowe}, pdfauthor = {P.Zając}]{hyperref}

%%%%%%

\author{Piotr Zając}
\shortauthor{P. Zając}


\titlePL{Ramię robota przemysłowego wspomagającego zadania montażowe}
\titleEN{Industrial Robotic Arm Supporting Assembly Tasks}


\shorttitlePL{Ramię robota przemysłowego wspomagającego zadania montażowe} % skrócona wersja tytułu jeśli jest bardzo długi
% \shorttitleEN{Preparation of a long and fascinating thesis in \LaTeX}


% Dopuszczalne wartości[1,2]:
% * "Projekt dyplomowy" - na koniec studiów I stopnia
% * "Praca dyplomowa" - na koniec studiów II stopnia
% [1] Zasady dyplomowania w roku akademickim 2020/2021 (Decyzja Dziekana WEAIiIB nr 16/2020 z dnia 9 grudnia 2020 roku)
% [2] Załącznik nr 1a) do Decyzji nr 16/2020 Dziekana Wydziału EAIiIB z dnia 09 grudnia 2020 r.
\thesistype{Projekt dyplomowy}
%\thesistype{Master of Science Thesis}

\supervisor{dr inż. Dawid Knapik}


\degreeprogramme{Automatyka i Robotyka}
%\degreeprogramme{Automatics and Robotics}

\date{2026}

% \department{Katedra Informatyki Stosowanej}
%\department{Department of Applied Computer Science}

\faculty{Wydział Elektrotechniki, Automatyki, Informatyki i Inżynierii Biomedycznej}
%\faculty{Faculty of Electrical Engineering, Automatics, Computer Science and Biomedical Engineering}

\acknowledgements{Serdecznie dziękuję XYZ\dots }


\begin{document}

	\titlepages
	\RedefinePlainStyle
	
	\setcounter{tocdepth}{2}
	\tableofcontents
	\clearpage

  
	\chapter{Wprowadzenie}
\label{cha:wprowadzenie}

Żyjemy w złotych czasach dla konstruktorów. Powszechny dostęp do darmowej wiedzy, narzędzi do obróbek różnorakich materiałów czy drukarek 3D sprawiają, że prototypowanie i wdrażanie swoich pomysłów w życie jest łatwiejsze niż kiedykolwiek.

Szczególnym osiągnięciem społeczności twórców są technologie open source, czyli bezpłatnie dostępne oprogramowania i projekty, które udostępniają kod źródłowy. Dzięki bezinteresownej pracy pasjonatów są one rozwijane i pozwalają niezależnym twórcom na stosowanie zaawansowanych narzędzi w swojej pracy.

Częstym tematem własnych projektów jest automatyzacja, przede wszystkim procesów, z którymi stykamy się w życiu codziennym. Automatyzacja coraz częściej pojawia się w naszych domach: inteligentni asystenci ułatwiający wyszukiwanie informacji za pomocą głosu, sztuczna inteligencja znacznie przyspieszająca przeprowadzanie żmudnych procesów, automatyczne narzędzia użytku domowego jak robot odkurzacz. Praca w warsztacie również została w znacznej części zautomatyzowana, prototypy są wytwarzane za pomocą druku 3D, natomiast części wykonane z metalu lub drewna coraz częściej nie są robione na tokarce ręcznej, tylko na maszynach CNC.

Jednak w kontekście automatyzacji domowej robot sześcioosiowy jest rzadko spotykany. Najtańsze roboty dostępne do kupienia, takie jak mechArm 270 czy  Dobot Nova 2 kosztują tysiące złotych i do ich obsługi potrzebne jest wstępne zaprogramowanie z poziomu komputera - umiejętność, którą nie każdy posiada.

Istnieją również ramienia open source, takie jak AR4 (https://anninrobotics.com/) czy też też BCN3D MOVEO, ale stworzenie samodzielnie takiego robota łączy się z wysokim kosztem i progiem umiejętności potrzebnym do jego wykonania. 

% cel pracy, Dlaczego temat jest istotny, Jakie problemy rozwiązuje robot? 
Celem projektu było zaprojektowanie i zbudowanie ramienia dostępnego cenowo, które miałoby znacznie ułatwiony mechanizm sterowania.


\subsection{Główne założenia projektu}
Wykonany w technologii druku 3D, modularny, przy użyciu technologii open source, umożliwia wykonywanie zadań konstrukcyjnych

%---------------------------------------------------------------------------

\section{Zawartość pracy}
\label{sec:zawartoscPracy}

W rozdziale~\ref{cha:podstawyTeoretyczne} przedstawiono podstawowe informacje na temat technologii użytych w projekcie.

W rozdziale~\ref{cha:mechanika} opisano przebieg projektowania konstrukcji robota, oraz rozwiązań z dziedziny mechaniki.

W rozdziale~\ref{cha:elektronika} przedstawiono schematy elektroniczne robota.

W rozdziale~\ref{cha:oprogramowanie} pokazane zostały rozwiązania oprogramowania robota.

W rozdziale~\ref{cha:testy} przedstawiono testy oraz analiza działania robota.


% Jeszcze kilka odnośników do bibliografii \cite{PeDa04,BuDo03}.
	\chapter{Podstawy Teoretyczne}
\label{cha:podstawyTeoretyczne}



\section{Aspekty prawne i bezpieczeństwo robotów wykonawczych}
\label{sec:kompilacja}
- technologia open-source
Projektu nie udałoby się wykonać, gdyby nie bezinteresowna praca wielu ludzi działających w projektach open source. 

- Normy bezpieczeństwa (np. ISO 10218).  
- Zagrożenia przy pracy robotów z ludźmi.  
- Odpowiedzialność prawna. 


\section{Sterowanie i algorytmy w robotyce}
\label{sec:narzedzia}

- Kinematyka prosta i odwrotna.  
- Sterowanie w przestrzeni współrzędnych.  
- Podstawy ROS2.  

\section{ROS2}
Opisać ROS2 dlaczego go wziąłem, co umożliwia. 
ROS2, czyli Robot Operating System to zestaw bibliotek oraz narzędzi znacznie ułatwiajacych pracę z projektami robotów. 




	\chapter{Mechanical design of the robot}
\label{cha:mechanika}
Mechanical design is a crucial part of any robotic system. It must account for the forces acting on the structure and be designed within the geometrical constraints. The design should balance performance with the cost of materials and the difficulty of manufacturing custom parts. An effective mechanical design also allows for easy assembly, maintenance, and future modifications, which is particularly important in prototype and research-oriented robots.

\section{Kinematic Structure and DOF Selection}
One of the first steps in the mechanical design process is selecting an appropriate kinematic structure for the intended application. That involves analyzing future workspace of the robot, types of operations it should be capable of performing,and the relative importance of factors such as speed, accuracy, and flexibility. 

The goal of this project is to develop a highly versatile robotic arm that can be adapted to a wide range of tasks rather than being optimized for a single application. This requirement strongly influences both the kinematic structure and the number of degrees of freedom of the robot.

\subsection{Degrees of freedom }
Robots movements are constrained by it's number of degrees of freedom (DOF). A robot arm with four degrees of freedom provides limited movability and is restricted to simple tasks where the position of tool center point (TCP) needs to be controlled, while the orientation is fixed. In contrast, a six-degree-of-freedom manipulator allows for full control of both position and orientation of the TCP in three-dimentional space.

Increasing the number of degrees of freedom introduces kinematic redundancy. While it gives the advantage in situations when it is required for the arm to avoid obstacles, it also increases mechanical complexity and computation of inverse kinematics is more demanding and requires using numerical methods.

For the purposes of this project, redundancy is not a strict requirement. A six-degree-of-freedom configuration provides sufficient dexterity and full control of the TCP while maintaining a reasonable balance between mechanical complexity, control effort, and versatility. Therefore, a 6DOF architecture was selected as the most suitable solution.

% \subsection{Common 6DOF Mechanical Architectures}

% After deciding to implement six degrees of freedom, the next step was to select an appropriate mechanical layout. Several 6DOF manipulator architectures are commonly used in industrial and research applications, each with its own advantages and limitations.

% A widely used solution is a serial kinematic chain, in which all joints are arranged sequentially. This configuration is mechanically simple and conceptually easy to design and model. However, one of the main challenges of a serial 6DOF robot is balancing actuator power against actuator mass. The second joint, typically corresponding to the shoulder pitch axis, is usually subjected to the highest torque, as it must support the weight of all subsequent links and actuators.

% To reduce the torque load on this joint, an offset shoulder design is often employed, where the actuator of the third joint is placed closer to the base, for example behind the first joint. This reduces the effective moment arm of the motor mass and improves the overall dynamic performance of the arm.

% Based on these considerations, a serial kinematic chain with an offset shoulder configuration was selected for this project, as it offers a good compromise between mechanical simplicity, load distribution, and achievable workspace.

\subsection{Final Selected Layout}

The final mechanical layout of the robot is based on an anthropomorphic structure, consisting of a shoulder, elbow, and wrist assembly. The orientation of the joint axes and the placement of actuators were chosen to achieve a balance between mechanical complexity, achievable dexterity, and ease of manufacturing.

% Example:
%   \nvar{\varx}{2cm}
\newcommand{\nvar}[2]{%
    \newlength{#1}
    \setlength{#1}{#2}
}

% Define a few constants for drawing
\nvar{\dg}{0.3cm}
\def\dw{0.25}\def\dh{0.5}
\nvar{\ddx}{1.5cm}

% Define commands for links, joints and such
\def\link{\draw [double distance=1.5mm, very thick] (0,0)--}
\def\joint{%
    \filldraw [fill=white] (0,0) circle (5pt);
    \fill[black] circle (2pt);
}
\def\grip{%
    \draw[ultra thick](0cm,\dg)--(0cm,-\dg);
    \fill (0cm, 0.5\dg)+(0cm,1.5pt) -- +(0.6\dg,0cm) -- +(0pt,-1.5pt);
    \fill (0cm, -0.5\dg)+(0cm,1.5pt) -- +(0.6\dg,0cm) -- +(0pt,-1.5pt);
}
\def\robotbase{%
    \draw[rounded corners=8pt] (-\dw,-\dh)-- (-\dw, 0) --
        (0,\dh)--(\dw,0)--(\dw,-\dh);
    \draw (-0.5,-\dh)-- (0.5,-\dh);
    \fill[pattern=north east lines] (-0.5,-1) rectangle (0.5,-\dh);
}

% Draw an angle annotation
% Input:
%   #1 Angle
%   #2 Label
% Example:
%   \angann{30}{$\theta_1$}
% \newcommand{\angann}[2]{%
%     \begin{scope}[red]
%     \draw [dashed, red] (0,0) -- (1.2\ddx,0pt);
%     \draw [->, shorten >=3.5pt] (\ddx,0pt) arc (0:#1:\ddx);
%     % Unfortunately automatic node placement on an arc is not supported yet.
%     % We therefore have to compute an appropriate coordinate ourselves.
%     \node at (#1/2-2:\ddx+8pt) {#2};
%     \end{scope}
% }

% Draw line annotation
% Input:
%   #1 Line offset (optional)
%   #2 Line angle
%   #3 Line length
%   #5 Line label
% Example:
%   \lineann[1]{30}{2}{$L_1$}
\newcommand{\lineann}[4][0.5]{%
    \begin{scope}[rotate=#2, blue,inner sep=2pt]
        \draw[dashed, blue!40] (0,0) -- +(0,#1)
            node [coordinate, near end] (a) {};
        \draw[dashed, blue!40] (#3,0) -- +(0,#1)
            node [coordinate, near end] (b) {};
        \draw[|<->|] (a) -- node[fill=white] {#4} (b);
    \end{scope}
}

% Define the kinematic parameters of the three link manipulator.
\def\thetaone{30}
\def\Lone{2}
\def\thetatwo{30}
\def\Ltwo{2}
\def\thetathree{30}
\def\Lthree{1}

\begin{tikzpicture}
    \robotbase
    \angann{\thetaone}{$\theta_1$}
    \lineann[0.7]{\thetaone}{\Lone}{$L_1$}
    \link(\thetaone:\Lone);
    \joint
    \begin{scope}[shift=(\thetaone:\Lone), rotate=\thetaone]
        \angann{\thetatwo}{$\theta_2$}
        \lineann[-1.5]{\thetatwo}{\Ltwo}{$L_2$}
        \link(\thetatwo:\Ltwo);
        \joint
        \begin{scope}[shift=(\thetatwo:\Ltwo), rotate=\thetatwo]
            \angann{\thetathree}{$\theta_3$}
            \lineann[0.7]{\thetathree}{\Lthree}{$L_3$}
            \draw [dashed, red,rotate=\thetathree] (0,0) -- (1.2\ddx,0pt);
            \link(\thetathree:\Lthree);
            \joint
            \begin{scope}[shift=(\thetathree:\Lthree), rotate=\thetathree]
                \grip
            \end{scope}
        \end{scope}
    \end{scope}
\end{tikzpicture}

% Define link lengths
\def\Lone{2}
\def\Ltwo{2}
\def\Lthree{1}
\def\Lfour{0.6}
\def\Lfive{0.4}

\begin{tikzpicture}

    % Base
    \robotbase

    % Joint 1
    \joint

    % Link 1
    \link(0:\Lone);
    \begin{scope}[shift=(0:\Lone)]
        \joint

        % Link 2
        \link(0:\Ltwo);
        \begin{scope}[shift=(0:\Ltwo)]
            \joint

            % Link 3
            \link(0:\Lthree);
            \begin{scope}[shift=(0:\Lthree)]
                \joint

                % Wrist joint 1 (J4)
                \link(0:\Lfour);
                \begin{scope}[shift=(0:\Lfour)]
                    \joint

                    % Wrist joint 2 (J5)
                    \link(0:\Lfive);
                    \begin{scope}[shift=(0:\Lfive)]
                        \joint

                        % Wrist joint 3 (J6) + gripper
                        \grip
                    \end{scope}
                \end{scope}
            \end{scope}
        \end{scope}
    \end{scope}

\end{tikzpicture}


\section{Actuator Selection and Torque Calculations}
The selection of the joint motors was based on availability, price, and dificulty to control.

For arm joints (joints 2 to 4) the motors of choice were servomotors Feetech FT5325M, capable of generating up to 25\,kg$\cdot$cm when powered at 7.4V.


The most critical joint from a torque perspective is the second joint (shoulder pitch). It needs to provide enough torque to lift the whole arm when it is in an extended position and have enough torque left to lift the held object. Even when the mass of the moved object is small, the length of the lever arm significantly increases the required torque. Therefore,the second arm motor used in the project was a higher-torque Feetech FI7635M that operates on the same voltage as the rest of the servomotors but can generate 35\,kg$\cdot$cm  of torque.

The first joint, responsible for base rotation, is actuated using a stepper motor. In this case, the torque requirements are related to the rotational inertia of the structure and the desired angular acceleration rather than gravitational loading. As a result, torque requirements for this joint are less restrictive compared to the shoulder and elbow joints. In addition, the axis is connected to the stepper motor shaft using 2GT belt and gears. If in the testing phase the stepper motor is lacking torque, the gear to gear ratio can be changed by increasing the number of teeth on the joint shaft gear. Another way of increasing torque is to limit the acceleration of the first joint.


The wrist joints (joints 5 and 6) are subjected to relatively low loads, as they only support the end effector and do not contribute significantly to lifting the arm. Therefore, micro servomotors Tower Pro MG-90S  were used for these joints, as their torque capabilities of 2,2 kg*cm are sufficient for the expected operating conditions.


\begin{table}[h]
    \centering
    \caption{Estimated minimum torque requirements for selected joints}
    \label{tab:torques}
    \begin{tabular}{|p{4cm}|p{5cm}|}
        \hline
        \textbf{Joint} & \textbf{Minimum torque required [kg$\cdot$cm]} \\
        \hline
        Joint 2 (Shoulder) & 10 \\
        \hline
        Joint 3 (Elbow) & 10 \\
        \hline
        Joint 4 (Wrist pitch) & 10 \\
        \hline
    \end{tabular}
\end{table}

The selected actuators provide a sufficient safety margin over the estimated minimum torque requirements, even in the worst-case configuration when the arm is fully extended. This margin compensates for modeling inaccuracies, dynamic effects, and additional loads.

\section{Material Selection and Manufacturing Methods}
3D printing technology has greatly improved in recent years and became an essential skill for modern engineers. It allows for prototyping parts in a short amount of time with good accuracy, and the cost of manufacturing parts is very low. This makes it perfect for the iterative design process of robotic systems.

\subsection{Manufacturing Approach}
Most common method of 3D printing is Fused Deposition Modeling (FDM). This method is easy to work with, low-cost, fast, and it allows for creating geometries impossible to manufacture using different methods like CNC machining. For the first iteration of the robot all mechanical parts were manufactured using FDM technology.
One of the parts printed for the robot was a GT2 pulley, if the accuracy of the FDM turned out to be insufficient, the pulley could be reprinted using Stereolithography (SLA). It uses resin as the print material, and can achieve layer heights as small as 0.025\,mm, resulting in improved surface finish and dimensional precision. 

\subsection{Material Comparison}
The filament used in FDM can be manufactured to have specific mechanical properties. It is crucial to choose the right filament type suitable for the project. Table~\ref{tab:filament_comparison} presents a comparison of commonly used FDM materials based on data summarized in an ALL3DP article~\cite{filaments}.

\begin{table}[h!]
    \centering
    \caption{Comparison of selected FDM filaments}
    \label{tab:filament_comparison}
    \begin{tabular}{|p{2cm}|p{6cm}|p{6cm}|}
        \hline
        \textbf{Material} & \textbf{Advantages} & \textbf{Disadvantages} \\
        \hline
        PLA & 
        Very easy to print, and thus great for prototyping; low warping; good dimensional accuracy; low cost. &
        Low temperature resistance; relatively low impact strength; not suitable for functional, load-bearing parts. \\
        \hline
        ABS &
        Higher mechanical strength than PLA; good impact resistance; moderate thermal resistance. &
        Difficult to print (warping, necessery to print in chamber because of fumes); lower dimensional accuracy than PLA. \\
        \hline
        PETG &
        Good balance of strength and printability; higher temperature resistance than PLA; minimal warping. &
        Slightly less rigid than PLA (problematic in robotics); lower impact resistance than ABS. \\
        \hline
        TPU &
        Flexible and elastic; great impact absorption; highly durable. &
        Difficult to print; not suitable for rigid structural components. \\
        \hline
        Nylon (PA) &
        Very strong and wear-resistant; excellent impact toughness; good thermal resistance. &
        High moisture absorption; warping risk; requires high printing temperature. \\
        \hline
        PC (Polycarbonate) &
        Extremely strong and heat-resistant; suitable for heavy-duty functional parts. &
        Very difficult to print - very high printing temperature and requires enclosure; demanding to tune and prone to warping. \\
        \hline
    \end{tabular}
\end{table}


\begin{figure}[h!]
    \caption{Filament types presented by their advantages and disadvantages\cite{filaments}}
    \centering
    \includegraphics[width=0.9\linewidth]{Images/filaments.png}
    % \label{fig:placeholder}
\end{figure}

\newpage

For the prototyping phase, PLA was the filament of choice because of it's ease of printing, low cost, and great dimentional accuracy. This allowed focusing on robot geometry and adjusting tolerances. In future iterations, the material could be changed with a more durable one such as PETG or ABS.

\subsection{Inserts, tolerances, and mechanical joints}
Although FDM 3D printing offers good dimensional accuracy, parts must be designed specifically with the limitations of the process in mind. During the design phase, large overhangs were avoided to reduce the need for support structures and to improve surface quality. For components requiring press-fit assembly, a dimensional offset of approximately 0.1\,mm was applied. Achieving reliable fits required several iterations of test prints and gradual adjustment of tolerances.


To ensure smooth rotation in joints, the parts are connected with ball bearings. This significantly reduced friction and wear compared to direct plastic-on-plastic contact. The bearings were mounted in the 3D printed parts using a press-fit assembly.

Rigid connections between 3D-printed parts can be achieved using press fits, adhesive bonding, or threaded fasteners. However, press fits alone limit serviceability, while self-tapping screws can damage the printed material over time. To ensure that the project can be assembled and disassembled without damaging the structure, brass threaded inserts were used. Inserts were heat-set into the printed components using a soldering iron, creating durable and reliable threaded connections.

To ensure proper retention of the inserts, the diameter of the printed holes was designed to be slightly smaller than the nominal insert diameter, allowing the surrounding plastic to melt and secure the insert during installation.


\subsection{Slicer Settings and Print Orientation}
An important aspect of FDM manufacturing is the orientation of parts on the print bed, as it directly determines layer direction and therefore influences both mechanical strength and dimensional accuracy. Aligning layers with the expected load paths was a key consideration during slicing.

Print orientation also affects the accuracy of cylindrical features. Holes printed with their axis aligned vertically typically exhibit higher roundness and dimensional precision compared to holes printed horizontally. For this reason, components containing bearing seats were oriented such that these features were printed from the top, even when this required additional support material.

\begin{figure}[h]
    \centering
    \includegraphics[width=0.9\linewidth]{Images/slicer.png}
    \caption{Positioning a part in slicer}
    \label{fig:slicer_positioning}
\end{figure}
\newpage
For components containing heat-set brass inserts, additional material was required around the inserts to prevent weakening of the structure. This was achieved by applying local modifiers in the slicer software to increase wall thickness around the insert holes.

\begin{figure}[h]
    \centering
    \includegraphics[width=0.75\linewidth]{Images/modifier.png}
    \caption{Local modifiers in the slicer}
    \label{fig:slicer_modifier}
\end{figure}

% should I describe infill?


\section{Base (First Joint) Mechanical Construction}
The design of the robot base posed one of the most significant mechanical challenges of the project. The base serves multiple functions: it houses the electronic components and provides adequate airflow for cooling them, ensures structural rigidity, allows convenient access for maintenance, and simultaneously acts as the first rotational joint of the robotic arm.

\subsection{Structural design}
To ensure ease of assembly and maintenance, the base was designed as a two-part structure connected using bolts and heat-set brass inserts. This modular approach allows the base to be opened repeatedly without damaging the printed material and enables easy use of the robot arm in a different configuration.

The lower part of the base serves as a structural foundation and can be rigidly mounted to a wooden or metal support plate. This improves overall stability and reduces vibrations during operation. The upper part of the base integrates the rotating elements of the first joint and supports the arm structure. 


\subsection{Electronics Integration} 
The lower section of the base contains a dedicated electronics mounting platform, conceptually similar to a computer motherboard tray. The custom control board and DC-DC buck converter are mounted using brass inserts to ensure mechanical stability and electrical insulation.

Electronic components such as the stepper motor driver and power converters generate heat during operation, which can negatively affect performance and reliability. To avoid this, forced airflow was implemented using a 12\,V DC fan mounted at the front of the base. Ventilation openings at the rear allow air to flow through the entire enclosure, effectively reducing the temperature inside.

\\
(zdjęcie części dolnej z cada oraz rzeczywiste)
\\
 

\subsection{First Axis Rotation Mechanism}
The first rotational axis connects the robotic arm to the base and must provide high rigidity while allowing smooth, low-friction motion. To achieve this, the joint was supported by two radial ball bearings, which ensure precise alignment and resistance to bending moments. Additionally, a needle bearing was used to carry axial loads resulting from the weight of the arm, preventing increased friction under vertical loading.

\\(image of the real bearings)\\

The first axis is driven by a stepper motor via a GT2 timing belt transmission. This solution allows for flexible torque adjustment through pulley ratio selection and isolates the motor from axial loads.

Additionally such design allowed for the implementation of a hollow rotating shaft. This enables routing of power and signal cables through the joint using a slip ring, eliminating cable twisting and allowing unlimited rotation of the base.

\begin{figure}[h]
    \centering
    \includegraphics[width=0.5\linewidth]{image.png}
    \caption{Cross-sectional view of the first axis assembly}
    \label{fig:first_axis_cross_section}
\end{figure}


\section{Arm Structure}
Using stepper motors in joints requires the implementation of external gear system to increase torque. This increases mechanical complexity, mass, and manufacturing effort, particularly when custom gears must be fabricated. To avoid this and reduce the mass of the arm, joints two through six were designed around standard-sized servomotors. These actuators incorporate an internal gear system that allows them to deliver high torque while remaining relatively small and lightweight. This approach simplified the mechanical design and reduced the number of custom components.

\subsection{Shoulder and Elbow Joint (second and third axis)}

\subsection{Wrist (Ases 4-6)}
This is important and deserves a subchapter:


\subsection{Cable Management Strategy}
To hide cables that power and control the motors inside the arm, the arm structure was designed with a tunnel that goes through the whole length, and between the parts the cables run through the inside of the ball bearing.

\begin{figure}
    \centering
    \includegraphics[width=0.8\linewidth]{Images/Arm2.png}
    \caption{Ramię 2}
    \label{fig:placeholder}
\end{figure}

\begin{figure}
    \centering
    \includegraphics[width=0.5\linewidth]{Images/cables_through_bearing.jpg}
    \caption{Enter Caption}
    \label{fig:placeholder}
\end{figure}

\section{End Effector (Wrist + Tools)}
Do obsługi ostatnich dwóch osi użyto lekkich serwomechanizmów małej mocy. Została ona zaprojektowana z myślą o łatwym dołączaniu różnego rodzaju narzędzi.

To move last two joints in the arm .

\subsection{TCP Reference Tool}
 W pierwszych testach zaprojektowano prosty element reprezentujący osie x, y oraz z. Ma on na celu wizualizację położenia TCP (tool center point).\\

\\(zdjęcie końcówki)\\


\section{Assembly, Tolerances, and Iteration Process}
How many iterations

\section{Summary of Mechanical Design}
CAD, real photos of final project
    \chapter{Projekt układów elektronicznych}
\label{cha:elektronika}


\section{ESP32 z microROS}
Microros pozwala na połączenie mikrokontrolera z układem ROS. Komunikacja przez UART lub Wifi.

\section{Mikroprocesow dla ROS2}

Sercem projektu miało być Raspberry Pi 5, posiadające wyjścia I/O umożliwiające komunikację mikrokontrolera z resztą elektroniki. \\
Nie udało się wykorzystać raspberry, pomijając, że jest słabe, nie pozwala ani sterować silnikiem krokowym, ani komunikować się sprawnie z microros w wersji jazzy, a humble nie da się zainstalować.

Dlatego mikroprocesowem całego projektu będzie laptop z Ubuntu 22.04 bootowanym z przenośnego dysku SSD.

\section{Serwonapędy oraz moduł PCA9685}
Do ruchu osiami zastosowano serwonapędy. \\
Krótki opis jak otrzymać wymagany kąt na serwonapędzie przy pwm\\
Raspberry pi nie jest dobre do generowania dobrego sygnału PWM, a co dopiero pięciu. Dlatego serwonapędy są sterowane przy pomocy modułu PCA9685, który generuje sygnały PWM na wszystkich serwonapędach jednocześnie, i komunikuje się z Raspberry Pi przez protokół I2C.
Dodatkowo pozwala to na przeprowadzenie przez slip ring jedynie 4 przewodów: 2 zasilające, 2 do I2C.

Serwonapędy osi 2, 3 i 4 są obsługiwane napięcie, 7,4V co pozwala uzyskać z nich maksymalny moment siły, natomiast ostatnie dwa serwonapędy są obsługiwane napięciem 6V. Dodatkowo, sterownik PCA9685 przyjmuje napięcie 5V. W takim wypadku, należy zamocować regulator napięcia Buck Converter, który zmniejsza napięcie z małymi stratami energii do 6V, natomiast do PCA9685 dodatkowo zamontować regulator LM7805 zmniejszający napięcie do 5V.

\section{Silnik krokowy}
Do poruszania pierwszą osią zastosowano silnik krokowy JK42HS48-1204, który kontrolowany jest przy pomocy sterownika serwonapędu DRV8825. 
Probem jest taki, że Raspberry pi nie nadaje się do generowania dobrego sygnału sterującego silnikiem krokowym, ponieważ raspberry pi z systemem linux nie jest systemem real time. Próby sterowania silnika krokowego przez Raspberry pi zakończyły się porażką, silnik wydawał tylko wysokie piskliwe dźwięki. 
W takich wypadkach, stosuje się połączenie mikroprocesora jakim jest raspberry z mikrokontrolerem. Do tego służy micro-ros. W moim projekcie użyję ESP32, potężnego mikrokontrolera zawierającego moduł bluetooth i wifi.

Stworzono płytkę prototypową w której połączono ESP ze sterownikiem. Sterownik należało nastroić do pracy z 12V (9V nie zadziałało) tak, żeby pomiędzy potencjometrem a masą było napięcie o wartości dwukrotnie niższej niż natężenie silnika. Wykonano testy, wszystko śmigało, następnie spróbowano wgrać microros i sterować silnikiem przez raspberry

Silnik krokowy wymaga bazowania przy starcie aby zapewnić poprawną orientację robota. Do tego wykorzystano magnes oraz cyfrowy czujnik halla A3144, który na wyjściu daje stan wysoki przy pojawieniu się pola magnetycznego. 

\section{Zasilanie całego projektu}
Ponieważ zastosowano silniki używające różnych napięć
\begin{itemize}
    \item Silnik krokody: 12V
    
\end{itemize}
\subsection{Przekształtnik buck}
Czym jest, jak działa, jaka sprawność, czy podbija natężenie. 

\subsection{Rozwiązanie zasilania}
(schemat blokowy) \\

(wstawić poprawione obliczenia)\\
Serwonapędy w najgorszym wypadku 3,5A, ale skoro pierwszy silnik się zatnie przy jakiejś masie to można policzyć ile wtedy będzie miało kolejne serwo, bo przecież ma o wiele krótsze ramię, i kolejne jeszcze krótsze, więc mogę oszacować ile w najgorszym przypadku będzie pobierać prądu. 
Mam 3 duże serwa pobierające 3,5A, dwa małe pobierające najgorzej 750mA, i silnik krokowy 1,2A. Więc najgorzej około 5,5A. Więc w najgorszym wypadku serwa będą pobierać 4,3A. Więc potrzebuję regulatora napięcia który zbije 12V do 7,4V do serwonapędów i wytrzyma 4,3A. Buck converter też zwiększy ampery. Więc 12V * xA = 7,4V x 12A, czyli x=7,4A, więc teoretycznie zasilacz 12V 8A jest jeszcze z małym zapasem.
Wniosek: Zasilacz 12V 10A powinien starczyć


    \chapter{Programming the Robot with ROS2}
\label{cha:oprogramowanie}
Using ROS2 in a project may take a longer time to set up and configure, but it has the advantage of a wide variety of ready to use tools that make the project easier in the long run.
ROS2 is an open source project and is still actively developed and changing, so choosing the right distribution should be considered at the beginning of the project to ensure access to the latest versions of tools. 

\section{ROS2 Environment}
ROS2 can run on a variety of Linux operating systems and even on Windows; however, it was developed with Linux Ubuntu in mind, and this operating system is recommended by the developers, so it was chosen for the project. 
Since it's release, a lot of ROS2 distributions have been created. I decided to use ROS2 Humble Hawksbill, not the latest LTS distro - that is Jazzy Jalisko, but Humble has been around for some time and I knew that I will find a lot of solutions for it on the internet. This distribution works on Ubuntu 22.04.
After setting up the operating system on a external SSD, I followed the instructions of setting up a ROS2 environment. 
(documentation) 
I tested the environment by running a turtlesim demo, that showcases the core logic of ROS. 

\section{Creating URDF}
Creating URDF file can done manually, but there are tools recommended by the ROS community \cite{creating_URDF} that can create URDF file directly from CAD model like Solidworks URDF exporter, or a plugin to Blender. In my project, because I was already using a Autodesk software for creating CAD models, I decided to use fusion2urdf\cite{fusion2urdf}.
To use the tool the CAD assembly model should be constructed following the rules included in the README of the repository, and the links named properly: using only small characters without space between them, and the first link has to be named base_link.
 
% Aby otrzymać bodies z części zaprojektowanych w inventorze, copy i paste samo body, a potem usuń plik. 


% Paczka jest generowana dla ROS1, dlatego kilka rzeczy trzeba poprawić. Folder z paczką bez spacji i małłymi literami.Dodaj do package.xml:
%   <depend>xacro</depend>
%   <depend>urdf</depend>
%   <depend>rviz</depend>
%   <depend>joint_state_publisher</depend>
%   <depend>joint_state_publisher_gui</depend>
%   <depend>robot_state_publisher</depend>

The add on generates a ROS1 package, and should be carefully remade for ROS2. That includes remaking the CMake file so it doesn't use catkin, creating a python launch file and the whole .rviz file. To avoid remaking the whole file manually, I used chat.gpt.

The advantage of using the package to generate the URDF file is that it includes STL of the model to it can be used in visualising with ROS, and it calculates moments of inertia.



\section{Visualising the robot with RViz}
To check if the generated URDF is correct, it could be visualised in RViz. It does not simulate the robot like Gazebo, but it is really useful for controling the state of the robot.

\noindent The following command enables generated file in RViz:
\begin{lstlisting}[language=bash]
  $ ros2 launch urdf_package display.launch.py
\end{lstlisting}
To check if the joints move correctly, another tool was used:
joint\_state\_ publisher. It simulates a publisher, that publishes joint states to the topic RViz is subscribed to.

\noindent The following command instals the joint state publisher:
\begin{lstlisting}[language=bash]
  $ ros2 launch urdf_package display.launch.py
\end{lstlisting}

RViz allows the user to visualise transforms of the robot and collision boxes.

screen z rviz i joint state publisher, potem też z transformatami.

\section{MoveIt2 Configuration}
To prepare the robot for MoveIt...

\subsection{Creating SRDF with MoveIt setup assistant}

After creating an URDF file of the robot, it is necessery to create a SRDF for MoveIt to work. It can be done with an official tool: Moveit Setup Assistant.

\href{https://moveit.picknik.ai/humble/doc/examples/ikfast/ikfast_tutorial.html
}{https://moveit.picknik.ai/humble/doc/examples/ikfast/ikfast_tutorial.html
}

\noindent Moveit setup assistant can be launched with the command: 

\begin{lstlisting}[language=bash]
  $ ros2 launch moveit_setup_assistant setup_assistant.launch.py
\end{lstlisting}

\begin{figure}
    \centering
    \includegraphics[width=0.5\linewidth]{Images/moveit_setup_assistant.png}
    \caption{Moveit Setup Assisnant window}
    \label{fig:placeholder}
\end{figure}

Opisać co się robi w setup assistant. 
Wygenerowanie macierzy kolizji
Virtual Joints
Stworzenie grup
End effector
ROS2_controller
Moveit2_controller


\noindent Po stworzeniu pliku SRDF należy sprawdzić jego poprawność w RViz, przy użyciu komendy: 

\begin{lstlisting}[language=bash]
  $ ros2 launch srdf2 demo.launch.py
\end{lstlisting}

\begin{figure}
    \centering
    \includegraphics[width=0.5\linewidth]{SRDF_RViz.png}
    \caption{Enter Caption}
    \label{fig:placeholder}
\end{figure}

W trakcie pracy natknąłem się na wiele błędów, przede wszystkim z tworzeniem URDF, a potem srdf przy pomocy narzędzia moveit. Do debugingu przydatne są komendy; 

\begin{lstlisting}[language=bash]
  $ ros2 control list_controllers
\end{lstlisting}
[INFO] [1763114714.374762157] [_ros2cli_7335]: waiting for service /controller_manager/list_controllers to become available...
joint_state_broadcaster joint_state_broadcaster/JointStateBroadcaster          active
david_arm_controller    joint_trajectory_controller/JointTrajectoryController  active


Należało stworzyć tylko jeden kontroler (do grupy david_arm) a nie dwa, jak automatycznie podpowiada program. W ros_controller należy usunąc wygenerowane dla grupy nadgarstka kontrolery.
ontroller name: david\_arm\_controller
Type: FollowJointTrajectory
Default: ✔
Joints:
  - joint1
  - joint2
  - joint3
  - joint4
  - joint5
  - joint6

  Nazwy nie mogą mieć spacji.
  % Teraz lecę na fake hardware, mozna to sprawdzić; 

\begin{lstlisting}[language=bash]
  $ ros2 control list_hardware_interfaces
\end{lstlisting}

Okazało się, że publishowałem w złej kolejności jointy, bo srdf czytał sobie stare bliki z folderów build i install, komenda zbawienna: 
\begin{lstlisting}[language=bash]
  $ ros2 topic echo /joint_states
\end{lstlisting}
Otrzymałem: 
header: stamp: sec: 1763107724 nanosec: 493116978 frame\_id: base\_link name: - joint2 - joint3 - joint1 - joint4 - joint5 - joint6 position: - 0.9101201539526196 - 1.1755129960850645 - -0.11931624533226025 - 1.5991373681522325 - 0.028240824535017816 - 0.1160348312634558 velocity: - 0.0 - 0.0 - 0.0 - 0.0 - 0.0 - 0.0 effort: - .nan - .nan - .nan - .nan - .nan - .nan ---

Widać tutaj, że kolejność jest zła pomimo poprawnego URDF. Ale sprawdziłem komendę: 

\begin{lstlisting}[language=bash]
  $ grep -R "joint_states" -n src/<your_moveit_pkg>
\end{lstlisting}

\subsection{OMPL (Open Motion Planning Library) }
How moveit calculates trajectory

\section{Hardware interface}

Aby umożliwić sterowanie silnikami, potrzebny jest mikrokontroler pozwalający na precyzyjne generowanie sygnałów. W projekcie zdecydowano się na ESP32, na którym zaimplementowano narzędzie microROS. Pozwala ono na zachowywanie się mikrokontrolera jak node, co wciela go do projektu ROS2. Na komputerze należy uruchomić tzw. microROS-agent, to on komunikuje się z mikrokontrolerem.

\subsection{Controlling servomotors through PCA9685}
PCA9685 module can be controlled with I2C protocol. On ESP32 any GPIO pin can be used to control I2C. Instead of using a library, I programmed simple API functions using registers. 
PCA9685 is configured during setup phase. 


\subsection{Driving the stepper motor with TMC2209 module}
In first iteration, a drv8855 module was used, but it was very loud when working. It was later switched to TMC2209 module. The pinout and control is the same so no changes needed to me made.

As oppose to a servomotor, stepper motor does not know it's current position, and can move only relatively an amount of steps. To solve it, the robot follows a homing procedure, where it rotates untill a hall sensor mounted on the base receives a magnetic pulse from a magnet located at the rotating part.


\subsection{Creating micro-ROS node on ESP32}
Cały kod trzeba będzie przerobić na ROS2
Aby uruchomić microRos należy najpierw na komputerze z linux postawić microros agent. 

Użyłem arduino IDE żeby zainstalować na esp32 kod micro ros, ale należało zwracać uwagę na spójnośc wersji board managera i biblioteki microrosarduino.\\
Przykładowo, w przypadku używania ROS2 w wersji Humble, 
W na początku komunikacja zadziałała, ale dodawanie liczb nie było płynne, dopiero po poprawie wersji biblioteki do wersji ROS2 humble.

Do micro_ros_agent w swoim workspace należy pobrać: 
git clone -b $ROS_DISTRO https://github.com/micro-ROS/micro_ros_setup.git src/micro_ros_setup

I zbudować projekt. 
Sprawdzono prosty kod wysyłający wartość zwiększającego się licznika, i odebrano na komputerze komendą: 
\noindent Komenda uruchamiająca wizualizację całego projektu :
\begin{lstlisting}[language=bash]
  $ ros2 run micro_ros_agent micro_ros_agent serial --dev /dev/ttyUSB0 -b 115200
\end{lstlisting}
 I sprawdzić publikowany topic:
ros2 topic echo /counter

Następnie stworzono subscriber, który przyjmuje topic z kątem serwomechanizmu, i przetestowano. 
ros2 topic pub /servo_angle std_msgs/msg/Float32 "{data: 45}"
Wszystko zadziałało. 

Testuj tym: 
ros2 topic pub /esp_joint_states sensor_msgs/msg/JointState \
"{name: ['joint1','joint2','joint3','joint4','joint5','joint6'], position: [0,0.2,0,0,0,0]}"

\section{Testing Hardware-Software connection}


% rqt_graph*

Sprawdzam 
ros2 topic info /joint_states
Otrzymuję: 
Type: sensor_msgs/msg/JointState
Publisher count: 1
Subscription count: 2

Screeny z moveit, jakie topiki i message, struktura message, czy trajektoria jest przesyłana

\section{Freedom of controlling the robot: voice control}
One of the most useful features of ROS is it's flexibility. When a robot is created, it can be controlled by a node that communicates with MoveIt2 API, and the commands can be sent by another node. If the method of controlling the robot changes, only the controlling node has to be changed, and it must send messages in the same format as before. For this project, the robot's TCP position was controlled with voice commands. 

\subsection{C++ API interface for MoveIt2}
I want to move the TCP: 
zajac@zajac-Nitro:~/Documents/David/Davie-Robot-Arm/david_ws
% $ ros2 run tf2_ros tf2_echo base_link tool_1
% [INFO] [1765277041.977375074] [tf2_echo]: Waiting for transform base_link ->  tool_1: Invalid frame ID "base_link" passed to canTransform argument target_frame - frame does not exist
% At time 1765277042.910531294
% - Translation: [-0.000, 0.095, 0.321]
% - Rotation: in Quaternion (xyzw) [-0.710, -0.000, -0.000, 0.704]
% - Rotation: in RPY (radian) [-1.579, -0.000, 0.000]
% - Rotation: in RPY (degree) [-90.453, -0.005, 0.004]
% - Matrix:
%   1.000  0.000 -0.000 -0.000
%   0.000 -0.008  1.000  0.095
%   0.000 -1.000 -0.008  0.321
%   0.000  0.000  0.000  1.000



\subsection{Voice recognition using Vosk library}
ROS2 nodes can be created in C++ or python, and the languages don't have to be the same between nodes - only the message type.
To show this feature, I created a Python node that uses a offline voice recognition toolkit: Vosk \cite{VOSK}, that would send messages to the C++ node tha communicates with MoveIt API.

Creating a python publisher is very simple, and only thing to look out for is to correctly send the message to a correct topic.

To communicate between nodes, I used a ready message type called Twist \cite{twist_message}. It consists from two vectors, one defining linear movement in three axes, and the second one the angular movement. 

After testing voice detection and defining commands responsible for movement in a given axis, the publisher was set to publish movement command to a topic.


\section{Summary of project software}
I created a node that does voice recognition, when it detects a command it sends a message containing movement. Another node, command2moveit subscribes to that topic and communicates with the MoveIt model to implement the command into robot movement. Then, a node moveit2microros takes positions from the simulated robot and sends them to the node on ESP32. That node is responsible for hardware interface.

\noindent To visualise the whole project structure, there is a ROS2 command:
\begin{lstlisting}[language=bash]
  $ ros2 run tf2_tools view_frames
\end{lstlisting}


\section{Symulacja działania robota w Gazebo}
Dobrą praktyką jest testowanie oprogramowania robota w symulacji przed wgraniem go na realny sprzęt. Pozwala to zapobiec nieoczekiwanym uszkodzeniom oraz przyspiesza proces testowania, przez możliwość szybkiej modyfikacji otoczenia oraz robota.\\
	\chapter{Testy}
\label{cha:testy}

\section{Działanie i testy robota}
\subsection{Sprawdzenie efektywności momentu siły pierwszej osi}
Po sprawdzeniu czy każda z osi poprawnie się porusza (nie wykryto żadnych oporów ruchu) eksperymentalnie sprawdzono momenty sił zawieszając obciążenia na robocie.

\subsection{Sterowanie silnika krokowego}
Stworzono płytkę prototypową z modułem DRV8855 i ESP32. 
Test1: 
Sterowanie silnika krokowego przy pomocy ESP32 zakończone sukcesem - płytka działa. 

Test2:
Sterowanie przez microROS.

\subsection{Sterowanie głosowe}
- Przykłady komend, czas reakcji, skuteczność rozpoznawania.  

\subsection{Analiza wyników}
- Testy poprawności działania.  
- Ocena stabilności manipulatora.  
- Ograniczenia projektu.  

\section{Możliwości dalszego rozwoju}
\subsection{Implementacja systemów wizyjnych (np śledzenie obiektu)}
\subsection{Sterowanie z poziomu aplikacji webowej}
\subsection{Integracja z dodatkowymi czujnikami}
\subsection{Rozszerzenie o systemy sztucznej inteligencji}
	\chapter{Podsumowanie}
\label{cha:posdumowanie}

\section{Wnioski}
- Podsumowanie zrealizowanych celów.  
- Ocena skuteczności przyjętych rozwiązań.  
- Refleksja nad ograniczeniami i możliwymi ulepszeniami.  

	
	% itd.
	% \appendix
	% \include{dodatekA}
	% \include{dodatekB}
	% itd.
	
	\printbibliography

\end{document}
