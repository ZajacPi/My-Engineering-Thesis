\chapter{Wprowadzenie}
\label{cha:wprowadzenie}

Żyjemy w złotych czasach dla konstruktorów. Powszechny dostęp do darmowej wiedzy, narzędzi do obróbek różnorakich materiałów czy drukarek 3D sprawiają, że prototypowanie i wdrażanie swoich pomysłów w życie jest łatwiejsze niż kiedykolwiek.

Szczególnym osiągnięciem społeczności twórców są technologie open source, czyli bezpłatnie dostępne oprogramowania i projekty, które udostępniają kod źródłowy. Dzięki bezinteresownej pracy pasjonatów są one rozwijane i pozwalają niezależnym twórcom na stosowanie zaawansowanych narzędzi w swojej pracy.

Częstym tematem własnych projektów jest automatyzacja, przede wszystkim procesów, z którymi stykamy się w życiu codziennym. Automatyzacja coraz częściej pojawia się w naszych domach: inteligentni asystenci ułatwiający wyszukiwanie informacji za pomocą głosu, sztuczna inteligencja znacznie przyspieszająca przeprowadzanie żmudnych procesów, automatyczne narzędzia użytku domowego jak robot odkurzacz. Praca w warsztacie również została w znacznej części zautomatyzowana, prototypy są wytwarzane za pomocą druku 3D, natomiast części wykonane z metalu lub drewna coraz częściej nie są robione na tokarce ręcznej, tylko na maszynach CNC.

Jednak w kontekście automatyzacji domowej robot sześcioosiowy jest rzadko spotykany. Najtańsze roboty dostępne do kupienia, takie jak mechArm 270 czy  Dobot Nova 2 kosztują tysiące złotych i do ich obsługi potrzebne jest wstępne zaprogramowanie z poziomu komputera - umiejętność, którą nie każdy posiada.

Istnieją również ramienia open source, takie jak AR4 (https://anninrobotics.com/) czy też też BCN3D MOVEO, ale stworzenie samodzielnie takiego robota łączy się z wysokim kosztem i progiem umiejętności potrzebnym do jego wykonania. 

% cel pracy, Dlaczego temat jest istotny, Jakie problemy rozwiązuje robot? 
Celem projektu było zaprojektowanie i zbudowanie ramienia dostępnego cenowo, które miałoby znacznie ułatwiony mechanizm sterowania.


\subsection{Główne założenia projektu}
Wykonany w technologii druku 3D, modularny, przy użyciu technologii open source, umożliwia wykonywanie zadań konstrukcyjnych

%---------------------------------------------------------------------------

\section{Zawartość pracy}
\label{sec:zawartoscPracy}

W rozdziale~\ref{cha:podstawyTeoretyczne} przedstawiono podstawowe informacje na temat technologii użytych w projekcie.

W rozdziale~\ref{cha:mechanika} opisano przebieg projektowania konstrukcji robota, oraz rozwiązań z dziedziny mechaniki.

W rozdziale~\ref{cha:elektronika} przedstawiono schematy elektroniczne robota.

W rozdziale~\ref{cha:oprogramowanie} pokazane zostały rozwiązania oprogramowania robota.

W rozdziale~\ref{cha:testy} przedstawiono testy oraz analiza działania robota.


% Jeszcze kilka odnośników do bibliografii \cite{PeDa04,BuDo03}.