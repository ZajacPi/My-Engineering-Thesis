\chapter{Kontstrukcja robota}
\label{cha:mechanika}

\section{Robot sześcioosiowy}
czym jest, dlazcego taki został zastosowany, schemat, przestrzeń robocza, 

\section{Druk 3D}
Dynamiczny rozwój technologii druku 3D w ostatnich latach umożliwia na wytwarzanie prototypowych części o wysokiej dokładności i złożoności w krótkim czasiem, jednocześnie niskim nakładem finansowym. Drukarki takie jak Bambu Lab A1 oferują bardzo wysoką dokładność oraz prędkość druku, automatyczną kalibrację oraz poziomowanie stołu a nawet zmianę filamentu przy użyciu dodatkowej stacji AMS, mieszcząc się w koszcie 1400zł. Tej drukarki użyłem w swoim projekcie. 
Części mechaniczne robota zostały wykonane głównie w technologii FDM (angl. Fused Deposition Modeling). Metoda ta poswala na szybkie i tanie prototypowanie, a także na stworzenie geometrii niemożliwej do wykonania w innych sposobach wytwarzania (np. obróbce CNC).  
Obecnie dostępny jest bardzo szeroki wybór rodzajów filamentów, z których najpopularniejsze to:

\begin{itemize}
 \item  PLA czyli  polilaktyd (ang. polylactic acid), łatwy w druku oraz tani, ale nie wykazuje dużej wytrzymałości oraz ma bardzo słabą odporność termiczną. 
 
 \item  ABS czyli akrylonitryl-butadien-styren (angl. Acrylonitrile butadiene styrene) mniej popularny od PLA oraz o wiele mocniejszy, ale trudniejszy w druku przez konieczność stosowania wyższej temperatury oraz używania komory do odprowadzania toksycznych oparów. 
 
 \item PETG  (angl. Polyethylene Terephthalate Glycol) czyli zmodyikowany PET znajdujący się w butelkach. Filament jest tworzony z recyklingowanych butelek, co sprawia, że używanie go jest dodatkowo przyjazne dla środowiska. Łączy zalety ABS oraz PLA. Jest łatwiejszy w druku od ABS, a jednocześnie oferuje zwiększoną wytrzymałość od PLA.
 \item TPU (Thermoplastic Polyurethane), czyli elastyczny filament 
 \item Nylon lub inaczej PA (angl. Polyamide),
 \item PC (angl. Polycarbonate) czyli jeden z najmocniejszych filamentów, odpornych na uszkodzenia mechaniczne oraz termiczne, jednak trudny w druku i wymagający dokładnych ustawień oraz komory.
\end{itemize}

\\
\begin{figure}
    \centering
    \includegraphics[width=0.5\linewidth]{Images/filaments.png}
    \caption{https://all3dp.com/1/3d-printer-filament-types-3d-printing-3d-filament/}
    \label{fig:placeholder}
\end{figure}

W procesie prototypowania aby skupić się na konstrukcji oraz poprawnych wymairach zastosowano PLA, przez co proces drukowania odbywał się bez problemów i efekty były bardzo zadowalające. 
W ostatecznej wersji planowana jest zmiana filamentu na bardziej odporny mechanicznie np. PETG lub ABS. 

Dokładne dopasowanie części tak, aby wchodziły na wcisk wymagało kilku iteracji testów konstrukcji. 
 

\section{Pierwsza oś}
Do ruchu pierwszej osi zastosowano silnik krokowy, który połączony jest z pierwszą osią za pomocą pasa transmisyjnego. Pozwala to na ewentualną zmianę przełożenia, oraz izoluje silnik od reszty konstrukcji.

Pierwsza oś montowana jest do podstawy za pomocą dwóch łożysk obrotowych co zapewnia sztywność, oraz łożyska igiełkowego oporowego zapewniającego dobry ślizg.

Aby zapobiec plątaniu się przewodów przy wielu obrotach zastosowano slip ring. 

\section{Kolejne osie}
Reszta osi jest poruszana przy pomocy serwomechanizmów o momencie siły 35kg*cm.

Największą obawą przy projektowaniu ramienia sześcioosiowego była druga oś, odpowiadająca za podnoszenie całego ramienia. Przez swoje długie ramię moment siły wywierany na serwomechanizm może być bardzo duży, i podnoszenie nawet małych ciężarów na pełnym wyciągnięciu mogłoby być wymagające. Jednak po wstępnym przeliczeniu momentów sił wywieranych przez poszczególne serwonapędy (największa masa ramienia) oraz wykonaniu i zważeniu wydruków 35kg*cm powinno zapewnić wystarczająco dużo siły. Jednak na wszelki wypadek ramię pierwsze zaprojektowano z możliwością dodania jeszcze jednego serwomechanizmu, co w efekcie podwoiłoby wywierany na pierwszą oś moment siły. 

Aby ukryć przewody w projekcie, zaprojektowano specjalne przejścia, a pomiędzy częściami przewody przechodzą przez łożyska.

\begin{figure}
    \centering
    \includegraphics[width=0.5\linewidth]{Images/Arm2.png}
    \caption{Enter Caption}
    \label{fig:placeholder}
\end{figure}


\section{Kiść}
Do obsługi ostatnich dwóch osi użyto małych serwomechnaizmów małęj mocy.

\\
testy momentu siły pojedynczego serwa, oraz momentu siły złożonego robota.


